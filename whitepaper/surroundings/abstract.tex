\begin{abstract}
    Currently, commonly used authentication methods are becoming obsolete,
    due to the lack of convenience, adversarial technology improvements
    or easy to make user errors.
    We believe that biometric authentication has the potential to become
    one of the safest, most convenient and most efficient methods.
    In this paper we describe our efforts to improve face recognition
    using a camera with depth-perception and infrared capabilities, as well as
    our search for new liveness detection methods such as skin detection
    using multispectral imaging.
    This text is mostly written with mobile solutions in mind,
    but the results could also be applied in other settings.
\end{abstract}

% Powszechnie używane dzisiaj metody uwierzytelniania stają się przestarzałe, co jest spowodowane niewygodą użycia, coraz lepszych metodach ich łamania, czy podatnością na błędy użytkowników. Uważamy, że zabezpieczenia biometryczne mają potencjał, żeby zostać jednymi z najbezpieczniejszych, najpowszechniejszych i najszybszych metod. W naszej pracy opisaliśmy nasze próby ulepszenia rozpoznawania twarzy przy użyciu kamer wykonujących zdjęcia głębi i podczerwieni, jak i szukaliśmy metod wykrywania życia takich jak wykrywanie skóry przy użyciu zdjęć multispektralnych. Ta praca została napisana z myślą o urządzeniach mobilnych, jednak jej wyniki mogą być wykorzystane w innych sytuacjach.
